% Created 2022-09-26 Mon 13:00
% Intended LaTeX compiler: pdflatex
\documentclass[11pt]{article}
\usepackage[utf8]{inputenc}
\usepackage[T1]{fontenc}
\usepackage{graphicx}
\usepackage{longtable}
\usepackage{wrapfig}
\usepackage{rotating}
\usepackage[normalem]{ulem}
\usepackage{amsmath}
\usepackage{amssymb}
\usepackage{capt-of}
\usepackage{hyperref}
\date{\today}
\title{}
\hypersetup{
 pdfauthor={},
 pdftitle={},
 pdfkeywords={},
 pdfsubject={},
 pdfcreator={Emacs 28.1 (Org mode 9.5.2)}, 
 pdflang={English}}
\begin{document}

\tableofcontents

\section{Definizione di:}
\label{sec:org5b0a8aa}
Piccola nota prima delle definizioni:

verso il 19xx ai matematici era partita una qualche ipocondria di
assiomatizzare tutto l'assiomatizzabile, mista a un minimalismo da far
sembrare tele bianche co' fossero capilettera minati.
(vedere Principia Matematica (quello di Russel e Whitehead) per un
esempio pratico, e vedersi il libro Goedel's Proof per il contesto
storico)

Potete immaginare come gli enti definiti in questi tempi fossero di un
astratto immane, a voler fare da denominatore comune a più o meno
tutta la matematica mai fatta fino ad allora, i gruppi/anelli
\textsc{et al} sono uno dei massimi esponenti di questo astrattismo da
Mondrian, come a voler ridurre tutta l'algebra di tutto (lineare,
reale, complessa, monomi, polinomii\ldots{}), la teoria degli insiemi,
magari la logica, e, perché no, il non toccare erba da anni, tutto a
una qualche \emph{res cogitans} comune, o cazzo si pippavano allora.

\subsection{Anelli e gruppi et al}
\label{sec:orgae7eebb}
\subsubsection{Anello}
\label{sec:orgaf9606b}
\[ (R,+,0_R;\circ ; 1_R) \]
Un insieme di roba tale che
\begin{itemize}
\item 1\textsubscript{R} e 0\textsubscript{R} \(\in\) R
\item Possiamo definire un'addizione tale che
\begin{itemize}
\item \(Coso + 0_R = Coso\) quando \(Coso \in R\)
\item La somma gode delle classiche proprietà
\begin{itemize}
\item \textbf{Commutativa}: \(a + b = b + a\) (quando \(a,b \in R\))
\item \textbf{Associativa}: \(a + (b + c) = (a + b) + c\) (quando \(a,b,c \in R\))
(che si definisce in questo modo perché la somma è un
operatore binario)
\end{itemize}
\item E di quelle meno classiche
\begin{itemize}
\item \textbf{Di gruppo}: \(\forall Coso \in R \to -Coso \in R\), dove
\(-Coso = \text{ quell'affare che } Coso + (-Coso) = 0_R\)
\end{itemize}
\end{itemize}
\item Possiamo definire un prodotto tale che
\begin{itemize}
\item \(Coso \circ 1_R = Coso\) quando \(Coso \in R\)
\item Il prodotto gode delle classiche proprietà
\begin{itemize}
\item \textbf{Associativa} \(a \circ (b \circ c) = (a \circ b) \circ c\)
\end{itemize}
\end{itemize}
\item La somma e il prodotto messi inseme godono delle classiche proprietà
\begin{itemize}
\item \textbf{Distributiva} \(a \circ (b + c) = (a \circ b) + (a \circ c)\)
\end{itemize}
\end{itemize}


\subsubsection{Anello Commutativo}
\label{sec:orgcc8b834}
Un anello per cui il prodotto ha anche proprietà commutativa, quindi
\[ (R,+,0_R;\circ ; 1_R) \]
tale che (Ctrl-C sezione sopra, Ctrl-V qui)
\begin{itemize}
\item Il prodotto gode anche dell proprietà
\begin{itemize}
\item \(a \circ b = b \circ a\) (quando \(a,b \in R\))
\end{itemize}
\end{itemize}
(lo definiamo come classe a parte perchè le matrici sono bambini
speciali che non voglioni avere un prodotto commutativo e vogliamo che
questa stra astrazione possa valere anche per quegli esseri)

\subsubsection{Gruppo commutativo}
\label{sec:orgdaf1845}
Prendi l'anello e ignora l'esistenza del prodotto e dell'\(1_R\), solo \(+\) e \(0_R\)

\subsubsection{Monoide unitario}
\label{sec:orge2c727d}
Prendi l'anello e ignora l'esistenza della somma e dello \(0_R\), solo
\(\circ\) e \(1_R\) (da cui prende il nome, credo)

\subsubsection{Morfismo di anello}
\label{sec:org04d2243}
Abbiamo due anelli \(a\) e \(b\), e una relazione \(\varphi\) : a \(\to\) b\footnote{è
abbstanza facile chiedersi a che cazzo serva una definzione del
genere, da quanto si vedrà quando diamo la definizione un morfismo di
anello è un qualcosa in cui posso prendere un teorema o un espressione
in anellese-a valido, passare tutto quello che riguarda l'anellaggine
in un morfismo \(\varphi\) : a \(\to\) b, e uscirne con un teorema o espressione in
anellese-b, anch'esso valido, più in genrale è una relazione tra a e b
che mantiene l'anellaggine, e ci interessa parecchio mantenere
l'anellagine, almeno in questa materia}
questa relazione si dice \emph{morfismo di anello} se mantiene
l'anellaggine di una relazione


\subsection{Equivalenza}
\label{sec:orgc2cc495}
\subsubsection{Definizione tirata}
\label{sec:org3373e2e}
Siano \(a\) e \(b\) due affari qualsiasi, facciamo due vettori.
\(a\) e \(b\) possono essere uguali in qualche modo, o avere qualche
caratteristica in comune, ad esempio possono avere entrabmi un primo
membro \(a_1 = b_1 = \text{ facciamo } 7\), se definiamo come relazione di
equivalenza \(a =_{\text{in questo caso}} b\) quando \(a_1 = b_1\) allora la classe
di equivalenza di \(a\) rispetto a questa relazione sarà l'insieme di tutti
i vettori \(b\) tali che \(a =_{\text{in questo caso}} b\).

Questo includerà (grazialcazzo) \(a\) visto che \(a_1 = a_1\) e in generale
\emph{tutti i vettori che condividono quella certa caratteristica con \(a\)}.
In pratica stiamo studiando l'apartheid di \(a_0\).

\section{Dimostrazione che}
\label{sec:orgf64b7b6}
\begin{enumerate}
\item le calssi di equivalenza di due elementi qualsiasi [x]\textsubscript{R} e [y]\textsubscript{R}
\end{enumerate}
\end{document}