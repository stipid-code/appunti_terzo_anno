% Created 2022-10-03 Mon 13:43
% Intended LaTeX compiler: pdflatex
\documentclass[11pt]{article}
\usepackage[utf8]{inputenc}
\usepackage[T1]{fontenc}
\usepackage{graphicx}
\usepackage{longtable}
\usepackage{wrapfig}
\usepackage{rotating}
\usepackage[normalem]{ulem}
\usepackage{amsmath}
\usepackage{amssymb}
\usepackage{capt-of}
\usepackage{hyperref}
\date{\today}
\title{}
\hypersetup{
 pdfauthor={},
 pdftitle={},
 pdfkeywords={},
 pdfsubject={},
 pdfcreator={Emacs 28.1 (Org mode 9.5.2)}, 
 pdflang={English}}
\begin{document}

\tableofcontents

\section{Definzione}
\label{sec:orgf982fa7}
\subsubsection{Nota storica, perchè esiste questa cosa?}
\label{sec:orgbda9592}
L'anello è stato introdotto per fare da denominatore comune a bene o
male ogni contesto in cui puoi fare somma e prodotto di qualcosa, cosa
comunque abbastanza utile quando sei nel 1900 e la gente sta provando
ad assiomatizzare e logicizzare ogni aspetto della matematica mai
concepito da essere vivente.

Quindi qualche decennio fa il genio di turno ha visto che c'erano
parecchie cose in comune tra interi, matrici, reali, complessi,
razionali e quant'altro, e ha deciso di definire questa \texttt{interface}
comune.

\subsection{Definzione a grandi linee}
\label{sec:org6398d93}
Un anello è un insieme per cui abbiamo definito le basi basi per
poterci fare calcoli, in questo caso
\begin{itemize}
\item Una somma (la sottrazione fai la somma con i negativi)
\item Un prodotto (la divisione fai il prodotto con l'inverso, se esiste)
\item Uno \(0\) che si comporta da \(0\)
(neutro per la somma e il prodotto fa \(0\))
\item Un \(1\) che si comporta da \(1\) (neutro per il prodotto)
\item Numeri positivi e negativi (i negativi servono per fare la sottrazione)
\end{itemize}

Per mettere un minimo di arrosto sotto a sto fumo prendiamo l'insieme
degli interi \(\mathbb{Z}\), tutto quello che abbiamo fatto con gli
interi e che sappiamo sugli interi ha come basi:
\begin{itemize}
\item Il fatto che lo \(0\) e l'\(1\) si comportano da \(0\) e \(1\)
\item Il fatto che abbiamo una somma con certe proprietà
\item Il fatto che abbiamo un prodotto con certe proprietà
\item Il fatto che se abbiamo \(x\) allora abbiamo anche \(-x\) e ci possiamo
fare le somme algebriche
\end{itemize}

\subsection{Definizione più rigorosa}
\label{sec:org120c1fa}
Un insieme
\[(R,0,+,1,\cdot)\]
è un anello quando
\begin{itemize}
\item \(0,1 \in R\)
\item la somma \(+\) è un operatore binario
\footnote{vuole dire solo che la somma è definita solo tra due elementi,
tanto quando sommi si fa la somma si fa comunque tra 2 addendi alla
volta, non cambia niente} 
da \(R \times R\)
\footnote{che ha in ingresso 2 elementi dell'anello}
in \(R\) che
\footnote{che ha come uscita un elemento dell'anello}
\begin{itemize}
\item è commutativo (\(a+b = b+a\) quando \(a,b \in R\))
\item è associativo (\(a+(b+c) = (a+b)+c\) quando \(a,b,c \in R\))
\end{itemize}
\item il prodotto \(\cdot\) è un operaTore binario da \(R\times R\) in \(R\) che
\begin{itemize}
\item è associativo \footnote{la proprietà commutativa del prodotto non vale
di default perchè altrimenti le matrici si arrabbiano, gli anelli
per cui la moltiplicazione è commutativa
(\( \mathbb{Z}, \mathbb{R}, \mathbb{C}, \mathbb{Q} \) \ldots{})
si dicono \emph{anelli commutativi} e faremo bene o male solo queli}
\end{itemize}
\item \(0\) è l'elemento neutro della somma (\(a+0 = a \forall a \in R\))
\item \(1\) è l'elemento neutro del prodotto
(\(a \cdot 1 = a \forall a \in R\))
\end{itemize}
\end{document}
