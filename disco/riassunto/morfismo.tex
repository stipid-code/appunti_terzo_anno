% Created 2022-10-03 Mon 08:31
% Intended LaTeX compiler: pdflatex
\documentclass[11pt]{article}
\usepackage[utf8]{inputenc}
\usepackage[T1]{fontenc}
\usepackage{graphicx}
\usepackage{longtable}
\usepackage{wrapfig}
\usepackage{rotating}
\usepackage[normalem]{ulem}
\usepackage{amsmath}
\usepackage{amssymb}
\usepackage{capt-of}
\usepackage{hyperref}
\date{\today}
\title{}
\hypersetup{
 pdfauthor={},
 pdftitle={},
 pdfkeywords={},
 pdfsubject={},
 pdfcreator={Emacs 28.1 (Org mode 9.5.2)}, 
 pdflang={English}}
\begin{document}

\tableofcontents

\section{Definizione}
\label{sec:org6906c29}
Un morfismo di anello è una funzione tra due anelli
\(\varphi : R \to S\) con la particolarità che tutte le "relazioni
anellose" tra   \(a,\ b,\ tizio,\ caio,\ sempronio,\text{et\ al}\ \in\ R\)
restano tali per \(\varphi(a),\ \varphi(b),\
\varphi(tizio),\ \varphi(caio),\ \varphi(sempronio),
\text{et}\ \varphi(\text{al})\ \in\ S\)

per definirlo in modo un po' più rigoroso diciamo che
\begin{itemize}
\item \(\varphi(\cdot)\) rispetta l'unità (\(\varphi(1_R) = 1_S\))
\item \(\varphi(\cdot)\) rispetta la somma (\(\varphi(a + _R b) = \varphi(a) + _S \varphi(b)\))
\item \(\varphi(\cdot)\) rispetta il prodotto (\(\varphi(a \cdot _R b) = \varphi(a) \cdot _S \varphi(b)\))
\end{itemize}

tutto quello che posso dire per \(m,n \in \mathbb{Z}\) rispetto a \(\mathbb{Z}\), ad esempio: 
\begin{itemize}
\item posso dirlo per \(\frac{m}{1}, \frac{n}{1}\) rispetto a \(\mathbb{Q}\)
\item posso dirlo per le matrici \(M = m \times I, N = n \times I\)
rispetto alle matrici quadrate in \(\mathbb{R}^{boh \times boh}\)
\item posso dirlo per i polinomii costanti \(m(x) = m, n(x) = n\) rispetto
all'insieme dei polinomii in \(x\)
\item etc\ldots{}
\end{itemize}
\end{document}